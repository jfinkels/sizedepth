%% sizedepth.tex - Size-depth tradeoffs for nondeterministic circuits
%%
%% Copyright 2014 Jeffrey Finkelstein.
%%
%% This LaTeX markup document is made available under the terms of the Creative
%% Commons Attribution-ShareAlike 4.0 International License,
%% https://creativecommons.org/licenses/by-sa/4.0/.
\documentclass{article}

%% This must come before hyperref.
\usepackage{amsthm}
%% This must come before hyperref.
\usepackage{thmtools}
%% This must come before complexity.
\usepackage{hyperref}
%% This is strongly recommended by biblatex.
\usepackage[english]{babel}
%% This is strongly recommended by biblatex.
\usepackage{csquotes}
\usepackage[backend=biber]{biblatex}
\usepackage{amsmath}
\usepackage{amssymb}
\usepackage{complexity}

\hypersetup{pdftitle={Size-depth-alternation tradeoffs for circuits}, pdfauthor={Jeffrey Finkelstein}}

\addbibresource{sizedepth.bib}

\declaretheorem[numberwithin=section]{theorem}
\declaretheorem[numberlike=theorem]{lemma}
\declaretheorem[numberlike=theorem]{corollary}
\declaretheorem[numberlike=theorem, style=definition]{definition}

\newcommand{\Q}{\mathsf{Q}}

\title{Size-depth-alternation tradeoffs for circuits}
\author{Jeffrey Finkelstein}

\begin{document}

\maketitle

A \emph{Boolean circuit} is a directed acyclic graph with some designated input gates of fan-in zero and one designated output gate of fan-out zero in which all non-input nodes are labeled with \textsc{or}, \textsc{and}, or \textsc{not}.
All \textsc{or} and \textsc{and} gates have fan-in two, and all \textsc{not} gates fan-in one.
We assume that the gates of a Boolean circuit are arranged in \emph{layers}; each layer consists of gates whose inputs come only from the previous layer and whose outputs feed only the following layer.
The \emph{depth}, or \emph{layer number}, of a gate is the length of the shortest path from any input node to the gate; the depth of the circuit is the depth of the output gate.
The input gates are in the first layer and the output gate is in the last layer.
We assume that the first layer after the input layer consists of only \textsc{not} gates, and no \textsc{not} gates appear anywhere else.

A \emph{configuration} of a Boolean circuit is a triple $(1^n, i, w)$, where $n$ is the number of inputs, $i$ is the layer number (in binary), and $w$ is the contents of the wires after layer $i$ (given as a single binary string).
If $C$ is a circuit, configuration $(1^n, i, u)$ \emph{yields} configuration $(1^n, i + 1, v)$ if layer $i + 1$ of the circuit produces output bits $v$ when provided with input bits $w$.
Configuration $(1^n, i, u)$ \emph{yields} configuration $(1^n, i + t, v)$ within $t$ steps of there is a sequence of binary strings $w_0, \dotsc, w_t$ such that
\begin{itemize}
\item $u = w_0$
\item $(1^n, i + k, w_k)$ yields $(1^n, i + k + 1, w_{k + 1})$ for all $k \in \{0, \dotsc, t - 1\}$
\item $w_t = v$
\end{itemize}

A family of circuits $\{C_n\}_{n \in \mathbb{N}}$ is $\L$-uniform if there is a Turing machine $M$ such that $M(1^n)$ halts within $O(\log n)$ space and outputs $C_n$.

\begin{definition}
  A language $L$ is decided by a $\Sigma_k$ circuit family $\{C_n\}_{n \in \mathbb{N}}$ if for any input string $x$,
  \begin{equation*}
    x \in L \iff \exists w_1 \forall w_2 \dotsb \Q_k w_k \colon C_n(x, w_1, \dotsc, w_k) = 1,
  \end{equation*}
  where $\Q_k$ is $\exists$ if $k$ is odd and $\forall$ if $k$ is even.
\end{definition}

\begin{definition}
  $\Sigma_k \SDG(s, d, g)$ is the class of languages that are decidable by an $\L$-uniform $\Sigma_k$ circuit family of size $s(n)$, depth $d(n)$, and $g(n)$ existential or universal bits at each quantifier.
\end{definition}

\begin{lemma}\label{lem:yields}
  Let $n$ be a natural number and let $A_n$ be a circuit on $n$ inputs of size $s$ and depth $d$.
  The problem of deciding whether configuration $C_i$ of circuit $A_n$ yields configuration $C_j$ within $t$ layers is in $\SD((n + s + \log d) + t(d + s), t + \log s)$.
\end{lemma}
\begin{proof}
  The circuit that decides this problem simulates $A_n$ for $t$ layers, then checks if any of those layers matches the target configuration $C_j$.
  Suppose $C_i = (1^n, i, w_i)$ and $C_j = (1^m, j, w_j)$.
  The circuit we construct will check that
  \begin{itemize}
  \item $n = m$
  \item $j < i + t$
  \item If layer $k$ of circuit $A_n$ is denoted $L_k$, then at least one of the layers $L_i, \dotsc, L_{i + t}$, when given $w_i$ as input, should output $w_j$
  \end{itemize}
  Assuming $m$ is in $O(n)$, checking the equality of $n$ and $m$ requires a circuit of size $O(n)$ and depth $O(\log n)$.
  TODO finish me\ldots
\end{proof}

An increased number of alternations can simulate a circuit that would normally have greater depth.
This is the circuit complexity analog of the corresponding result for Turing machines, described in \autocite[Lemma~2.7]{williams06}, therein attributed to \autocite[Theorem~5.1]{fvm00}.
(The $\Sigma_k$ algorithm in those works is given as a recursive alternating Turing machine algorithm; in this work, we have unwrapped the recursion so that it more closely matches the definition of $\Sigma_k \SDG$.)

\begin{lemma}
  For all $b$, $k$, $s$, and $d$,
  \begin{equation*}
    \SD(s, d) \subseteq \Sigma_k \SDG(s', d', g') % \cap \Pi_k \SDG(s', d', g')
  \end{equation*}
  where
  \begin{itemize}
  \item $s' = k b (n + s + \log d) + \frac{d}{b^{k - 1}}(d + s)$
  \item $d' = O(\max(\log b k (n + s + \log d), \frac{d}{b^{k - 1}} + \log s))$
  \item $g' = b (n + s + \log d)$
  \end{itemize}
\end{lemma}
\begin{proof}
  Let $L$ be a language in $\SD(s, d)$, and let $\{A_n\}_{n \in \mathbb{N}}$ be the $\L$-uniform circuit family that decides it.
  Suppose $M$ is the $O(\log n)$ space Turing machine that outputs $A_n$ on input $1^n$.
  First we show a $\Sigma_{2k}$ circuit family that decides $L$, then we modify it to provide a $\Sigma_k$ circuit family.

  Let $n$ be an arbitrary natural number and let $x$ be a binary string of length $n$.
  Suppose $C_0$ is the initial configuration of $A_n$ with input $x$ and $C_{d + 1}$ is the accepting configuration.
  Consider the $\Sigma_{2k}$ formula
  \begin{equation}\label{eq:2k}
    \exists C^k \forall r_k \dotsb \exists C^1 \forall r_1 \colon B_n(x, C^k, \dotsc, C^1, r_k, \dotsc, r_1) = 1,
  \end{equation}
  where each $C^i$ is a vector of $b$ circuit configurations and each $r_i$ is a binary integer in $\{1, \dotsc, b\}$.
  The Boolean circuit $B_n$ computes the conjunction of the following three formulas.
  \begin{align}
    & C_0 = C^k_1 \land C^k_b = C_{d + 1} \label{frm:1} \\
    & \bigwedge^k_{i = 2} \left(C^i_{r_i} = C^{i - 1}_1 \land C^{i - 1}_b = C^i_{r_i + 1} \right) \label{frm:2} \\
    & C^1_{r_1} \text{ yields } C^1_{r_1 + 1} \text{ within } \frac{d}{b^{k - 1}} \text{ layers} \label{frm:3}
  \end{align}
  The first formula ensures that the initial configuration and the accepting configuration of the circuit are correctly guessed as the first and last configurations.
  The second formula ensures that the guessed sequence of configurations is a valid sequence.
  The third formula ensures that one of the guessed sequences of configurations correctly computes the original circuit $A_n$.
  Let us prove that this is an \L-uniform $\Sigma_{2k} \SDG(s', d', g')$ circuit family that decides the language $L$.

  First, we must prove \L-uniformity.
  (TODO show \L-uniformity\ldots)

  Next, let's determine the size and depth of the circuit.
  Each configuration requires $O(n + s + \log d)$ bits to describe; let $S$ denote this value.
  Selecting $C^i_{r_i}$ from the sequence $C^i$ requires a multiplexer in which each input has $O(S)$ bits, there are $b$ such inputs, and the selector $r_j$ has $\left\lceil\log b\right\rceil$ bits.
  This can be implemented by $O(S)$ single-bit multiplexers working in parallel.
  A multiplexer for $b$ input bits can be implemented by a circuit of size $O(b)$ and depth $O(\log b)$ \autocite[Lemma~2.5.5]{savage98}.
  Thus the full multiplexer requires size $O(b S)$ and depth $O(\log b)$.

  Formula~\eqref{frm:1} contains two equality checks and requires two multiplexers.
  Checking equality of two configurations requires a tree of \textsc{and} gates of size $O(S)$ and depth $O(\log S)$.
  It can therefore be implemented with a circuit of size $O(b S)$ and depth $O(\log b + \log S)$, or more simply $O(\log b S)$.

  Formula~\eqref{frm:2} contains $O(k)$ equality checks, multiplexers, and increments of $r_i$.
  Computing $r_i + 1$ from $r_i$ requires a circuit of size $O(\log b)$ and depth $O(\log \log b)$.
  Each equality check, multiplexer, and increment can be performed in parallel, and the conjunction of these $O(k)$ bits can be computed by a tree of \textsc{and} gates of size $O(k)$ and depth $O(\log k)$.
  Therefore, this formula can be implemented with a circuit of size $O(k b S)$ and depth $O(\log b + \log S + \log k)$, or more simply $O(\log b k S)$.
  
  By \autoref{lem:yields}, formula~\eqref{frm:3} can be implemented by a circuit of size $O(S + \frac{d}{b^{k - 1}}(d + s))$ and depth $O(\frac{d}{b^{k - 1}} + \log s)$.
  Combining the three formulas with a conjunction requires only $O(1)$ more \textsc{and} gates so the overall size of the circuit $B_n$ is the sum of the sizes of the circuits computing the three formulas above, and the depth of the circuit is the maximum of the three depths.
  Therefore, the size of $B_n$ is $O(k b S + \frac{d}{b^{k - 1}} (d + s))$ and the depth is $O(\max(\log b k S, \frac{d}{b^{k - 1}} + \log s))$.
  Furthermore, each existential quantifier guesses $O(b S)$ bits and each universal quantifier guesses $O(\log b)$ bits, so the number of bits guessed at each quantifier is bounded by $O(b S)$.

  (TODO show correctness\ldots)

  Now we must eliminate half the alternations in formula~\eqref{eq:2k}.
  To do this, we ``negate'' every other pair of quantifiers, along with the corresponding equality checks in the circuit $B_n$.
  (TODO explain why this is possible\ldots)
  Two quantifiers of the same type that appear next to one another can be collapsed to a single quantifier of that type; the number of bits guessed at each quantifier remains $O(b S)$.
  So in the case that $k$ is even, formula~\eqref{eq:2k} becomes
  \begin{equation}\label{frm:k}
    \exists C^k \forall r_k \forall C^{k - 1} \exists r_{k - 1} \dotsb \exists C^2 \forall r_1 \forall C^2 \exists r_1 \colon B'_n(x, C^k, \dotsc, C^1, r_k, \dotsc, r_1) = 1,
  \end{equation}
  where $B'_n$ represents $B_n$ with the following two modifications.
  First, whenever $i$ is even in formula~\eqref{frm:2}, replace $C^{i - 1}_b = C^i_{r_i + 1}$ with its negation.
  Second, if $k$ is even, replace formula~\eqref{frm:3} with its negation.
  These modifications will not affect the size or depth of the circuit, except by small constant factors.
  Thus formula~\eqref{frm:k} represents a $\Sigma_k \SDG(s', d', g')$ circuit that correctly decides $L$, concluding the proof.
\end{proof}

Our goal is to choose $b$ in order to minimize the depth $d'$ of the $\Sigma_k$ circuit.
We wish to minimize $d'$, so we consider $\log k b S$ and $\frac{d}{b^{k - 1}} + \log s$ as functions of $b$, where $S = s + n + \log d$ as in the previous lemma.
The former function is monotonically increasing and the latter monotonically decreasing.
The smallest value of the maximum of the two functions is achieved when the two functions have the same value.
Solving the equation $\log k b S = \frac{d}{b^{k - 1}} + \log s$ for $b$ yields the equation $b = \ldots$.
This choice of $b$ yields complicated expressions for the values of $s'$, $d'$, and $g'$, and is ultimately uninformative (to me).
In order to find a simpler expression for $d'$, we relax the depth $d'$ from the maximum to the sum.

\begin{corollary}
  For all $k$, $s$, and $d$,
  \begin{equation*}
    \SD(s, d) \subseteq \Sigma_k \SDG(s', d', g') % \cap \Pi_k \SDG(s', d', g')
  \end{equation*}
  where
  \begin{itemize}
  \item $s' = k S (d (k - 1))^{\frac{1}{k - 1}} + \frac{d + s}{k - 1}$
  \item $d' = \log\left(k S (d (k - 1))^{\frac{1}{k - 1}}\right) + \frac{1}{k - 1} + \log s$
  \item $g' = S (d (k - 1))^{\frac{1}{k - 1}}$
  \end{itemize}
\end{corollary}
\begin{proof}
  Let $S = n + s + \log d$ as in the previous lemma.
  We know that
  \begin{equation*}
    \max(\log k b S, \frac{d}{b^{k - 1}} + \log s) \leq \log k b S + \frac{d}{b^{k - 1}} + \log s
  \end{equation*}
  as long as all the variables are positive, so any circuit with depth given by the left side of the inequality can be simulated by a circuit with depth given by the right side (by padding with the appropriate number of layers computing the identity function).
  If we view $\log k b S + \frac{d}{b^{k - 1}} + \log s$ as a function of $b$, the minimum is achieved when $\frac{1}{b} + d (1 - k) b^{-k}$, the first derivative respect to $b$, is zero.
  Since $b$ is positive, this occurs exactly when $b = (d (k - 1))^{\frac{1}{k - 1}}$.
  This setting of $b$ yields the following values for $s'$, $d'$, and $g'$.
  \begin{itemize}
  \item $s' = k S (d (k - 1))^{\frac{1}{k - 1}} + \frac{d + s}{k - 1}$
  \item $d' = \log\left(k S (d (k - 1))^{\frac{1}{k - 1}}\right) + \frac{1}{k - 1} + \log s$
  \item $g' = S (d (k - 1))^{\frac{1}{k - 1}}$ \qedhere
  \end{itemize}
\end{proof}

\begin{corollary}
  \begin{equation*}
    \P \subseteq \Sigma_2 \SDG(\poly, \log n, \poly).
  \end{equation*}
\end{corollary}
\begin{proof}
  $\P$ is exactly the class of languages decidable by $\L$-uniform polynomial size circuits \autocite[Theorem~6.15]{ab09}.
  In other words, $\P = \SD(\poly, \poly)$.
  By the previous corollary, $\SD(\poly, \poly) \subseteq \Sigma_k (\poly, \log, \poly)$ for any $k \geq 2$.
  Thus $\P \subseteq \Sigma_2 \SDG(\poly, \log n, \poly)$.
\end{proof}

Unfortunately, this corollary is weaker than already known results:
\begin{equation*}
  \P \subseteq \NP \subseteq \NNC^1(\poly) \subseteq \Sigma_1 \SDG(\poly, \log n, \poly) \subseteq \Sigma_2 \SDG(\poly, \log n, \poly)
\end{equation*}
(The second inclusion comes from \autocite[Theorem~2.2]{wolf94}.)

\printbibliography

\end{document}
